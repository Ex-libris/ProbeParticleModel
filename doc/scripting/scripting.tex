\section{Examples}
Example of python scripts which setup simulation, run it, and plot results are
can be seen in test2.py and testServer2.py files.

Here we provide overview of various operations done in this example scripts with
a bit more detailed explanation than just inline-comments within the script
code. The topics and are discussed in order in which are done in the example the
script test2.py. The \textbf{order} of operations is sometimes important ( e.g. you
cannot allocate grid until you set it's size, you cannot do scan until you
sample forcefiled etc. ) 


\subsection{Import common libraries}

\begin{shadedbox}
    \begin{lstlisting}[language=python]
import os
import numpy as np
import matplotlib.pyplot as plt
import elements
import basUtils
    \end{lstlisting}
\end{shadedbox}




\subsection{Load ProbeParticle library}

C++ part of ProbeParticle library ( ProbeParticle.cpp ) is compiled into
ProbeParticle\_lib.so binary dynamic library. If we do any change in C++ code, we
have to recompile the library. This is done automatically during call of import
ProbeParticle if the file ProbeParticle\_lib.so is not pressent in the directory.
So, If we want force recompilation of the C++ dynamic library, we just delelete
the library. This can be done e.g. like this: 


\begin{shadedbox}
    \begin{lstlisting}[language=python]
def makeclean( ):
    import os
    [ os.remove(f) for f in os.listdir(".") if f.endswith(".so") ]
    [ os.remove(f) for f in os.listdir(".") if f.endswith(".o") ]
    [ os.remove(f) for f in os.listdir(".") if f.endswith(".pyc") ]

makeclean( )  # force to recompile 
import  ProbeParticle as PP

    \end{lstlisting}
\end{shadedbox}



\subsection{Setup system and simulation parameters}

One way to change simulation parameters is to change the default dictionary
PP.params. Which is by default e.g. defined like this:


\begin{shadedbox}
    \begin{lstlisting}[language=python]
params={
'PBC': False,
'gridN':       np.array( [ 150,     150,   50   ] ).astype(np.int32),
'gridA':       np.array( [ 12.798,  -7.3889,  0.00000 ] ),
'gridB':       np.array( [ 12.798,   7.3889,  0.00000 ] ),
'gridC':       np.array( [      0,        0,      5.0 ] ),
'moleculeShift':  np.array( [  0.0,      0.0,    -2.0 ] ),
'probeType':   8,
'charge':      0.00,
'r0Probe'  :  np.array( [ 0.00, 0.00, 4.00] ),
'stiffness':  np.array( [ 0.5,  0.5, 20.00] ),
'scanStep': np.array( [ 0.10, 0.10, 0.10] ),
'scanMin': np.array( [   0.0,     0.0,    5.0 ] ),
'scanMax': np.array( [  20.0,    20.0,    8.0 ] ),
'kCantilever'  :  1800.0, 
'f0Cantilever' :  30300.0,
'Amplitude'    :  1.0,
'plotSliceFrom':  16,
'plotSliceTo'  :  22,
'plotSliceBy'  :  1,
'imageInterpolation': 'nearest',
'colorscale'   : 'gray',
}
   \end{lstlisting}
\end{shadedbox}

An alternative way is to load the parameters from a file using PP.loadParams.
This should be done before all other operations ( like definition and allocation
of sampling and scanning grid )




\begin{shadedbox}
    \begin{lstlisting}[language=python]
PP.loadParams( 'params.ini' ) # load parametes from ini file
   \end{lstlisting}
\end{shadedbox}


Then we should define system geometry ( positions of atoms and it types )
\begin{shadedbox}
    \begin{lstlisting}[language=python]
atoms    = basUtils.loadAtoms('input.xyz', elements.ELEMENT_DICT )
Rs       = np.array([atoms[1],atoms[2],atoms[3]]);  
iZs      = np.array( atoms[0])
   \end{lstlisting}
\end{shadedbox}




ize and shape of sampling and scanning grid is normally set by parameters
PP.params['gridA'],'gridB','gridC','scanMin','scanMax'. However, in case of
non-periodic samples ( such as single molecule ) it is more conveninent let
program build proper sampling and scanning box around the molecule, instead of
defining the super-lattice verctors. This can be done by calling PP.autoGeom(
fitCell=True ) function like this:

Before the simulation the geometry of molecule should be shifted to proper
position with respect to sampling grid. The shift of the molecule can be set
either manually in PP.params['moleculeShift' ], or let program set this
parameter automatically using PP.autoGeom( shiftXY=True ). 

        
        
\begin{shadedbox}
    \begin{lstlisting}[language=python]
if not PP.params['PBC' ]:
    PP.autoGeom( Rs, shiftXY=True,  fitCell=True,  border=3.0 )

Rs[0] += PP.params['moleculeShift' ][0]          # shift molecule so that we sample reasonable part of potential 
Rs[1] += PP.params['moleculeShift' ][1]          
Rs[2] += PP.params['moleculeShift' ][2]          
Rs     = np.transpose( Rs, (1,0) ).copy() 

   \end{lstlisting}
\end{shadedbox}

If we used periodic boundary condition, it is necessary to multiply geometry of
sample to neighboring unit cells. There is automatic procedure PP.PBCAtoms() to
do that:

\begin{shadedbox}
    \begin{lstlisting}[language=python]
if PP.params['PBC' ]:
    iZs,Rs,Qs = PP.PBCAtoms( iZs, Rs, Qs, avec=PP.params['gridA'],
    bvec=PP.params['gridB'] )
   \end{lstlisting}
\end{shadedbox}


Beside the geometry (position of atoms) the sample properties are defined also
by parameters of this atoms,such as Lenard-Jones parameters and charge. In
principle it is possible to set parameters (C6,C12 and Q ) of each atom
independently by hand (it is just array of numbers). However, more convenient
way is to read it from file of L-J parameters by PP.loadSpecies() for each atom
type and set the the parameters for each atom instance by PP.getAtomsLJ().
Charges for pairwise pointcharge Coulomb interaction are read from 5-th column
of geometry file.

\textbf{NOTE}: it is important to do this step after the multiplication of periodic
images to neighboring cells. 

\begin{shadedbox}
    \begin{lstlisting}[language=python]
Qs       = np.array( atoms[4] )
FFparams = PP.loadSpecies        ( 'atomtypes.ini'  )
C6,C12   = PP.getAtomsLJ( PP.params['probeType'], iZs, FFparams )
   \end{lstlisting}
\end{shadedbox}

\subsection{Define and allocate arrays}
Do this before simulation, in case it will crash

\begin{shadedbox}
    \begin{lstlisting}[language=python]
dz    = PP.params['scanStep'][2]
zTips = np.arange( PP.params['scanMin'][2], PP.params['scanMax'][2]+0.00001, dz )[::-1];
ntips = len(zTips); 
print " zTips : ",zTips
rTips = np.zeros((ntips,3))
rs    = np.zeros((ntips,3))
fs    = np.zeros((ntips,3))

rTips[:,0] = 1.0
rTips[:,1] = 1.0
rTips[:,2] = zTips 

PP.setTip()

xTips  = np.arange( PP.params['scanMin'][0], PP.params['scanMax'][0]+0.00001, 0.1 )
yTips  = np.arange( PP.params['scanMin'][1], PP.params['scanMax'][1]+0.00001, 0.1 )
extent=( xTips[0], xTips[-1], yTips[0], yTips[-1] )
fzs    = np.zeros(( len(zTips), len(yTips ), len(xTips ) ));

nslice = 10;

FFparams = PP.loadSpecies        ( 'atomtypes.ini'  )
C6,C12   = PP.getAtomsLJ( PP.params['probeType'], iZs, FFparams )

print " # ============ define Grid "

cell =np.array([
PP.params['gridA'],
PP.params['gridB'],
PP.params['gridC'],
]).copy() 

gridN = PP.params['gridN']

FF   = np.zeros( (gridN[2],gridN[1],gridN[0],3) )

   \end{lstlisting}
\end{shadedbox}




\subsection{Sample Lenard-Jones and electrostatic potential}

\begin{shadedbox}
    \begin{lstlisting}[language=python]

PP.setFF( FF, cell  )
PP.setFF_Pointer( FF )
PP.getLenardJonesFF( Rs, C6, C12 )

plt.figure(figsize=( 5*nslice,5 )); plt.title( ' FF LJ ' )
for i in range(nslice):
    plt.subplot( 1, nslice, i+1 )
    plt.imshow( FF[i,:,:,2], origin='image', interpolation='nearest' )


withElectrostatics = ( abs( PP.params['charge'] )>0.001 )
if withElectrostatics: 
    print " # =========== Sample Coulomb "
    FFel = np.zeros( np.shape( FF ) )
    CoulombConst = -14.3996448915;  # [ e^2 eV/A ]
    Qs *= CoulombConst
    #print Qs
    PP.setFF_Pointer( FFel )
    PP.getCoulombFF ( Rs, Qs )
    plt.figure(figsize=( 5*nslice,5 )); plt.title( ' FFel ' )
    for i in range(nslice):
        plt.subplot( 1, nslice, i+1 )
        plt.imshow( FFel[i,:,:,2], origin='image', interpolation='nearest' )
    FF += FFel*PP.params['charge']
    PP.setFF_Pointer( FF )
    del FFel

plt.figure(figsize=( 5*nslice,5 )); plt.title( ' FF total ' )
for i in range(nslice):
    plt.subplot( 1, nslice, i+1 )
    plt.imshow( FF[i,:,:,2], origin='image', interpolation='nearest' )
   \end{lstlisting}
\end{shadedbox}

\subsection{3D-Scan with ProbeParticle relaxation}
\begin{shadedbox}
    \begin{lstlisting}[language=python]
for ix,x in enumerate( xTips  ):
    print "relax ix:", ix
    rTips[:,0] = x
    for iy,y in enumerate( yTips  ):
        rTips[:,1] = y
        itrav = PP.relaxTipStroke( rTips, rs, fs ) / float( len(zTips) )
        fzs[:,iy,ix] = fs[:,2].copy()
    \end{lstlisting}
\end{shadedbox}



\subsection{Convert Fz -> df}
\begin{shadedbox}
    \begin{lstlisting}[language=python]
dfs = PP.Fz2df( fzs, dz = dz, k0 = PP.params['kCantilever'], f0=PP.params['f0Cantilever'], n=int(PP.params['Amplitude']/dz) )
    \end{lstlisting}
\end{shadedbox}



\subsection{Plot results of relaxed 3D-scan}
\begin{shadedbox}
    \begin{lstlisting}[language=python]
print " # ============  Plot Relaxed Scan 3D "

#slices = range( PP.params['plotSliceFrom'], PP.params['plotSliceTo'], PP.params['plotSliceBy'] )
#print "plotSliceFrom, plotSliceTo, plotSliceBy : ", PP.params['plotSliceFrom'], PP.params['plotSliceTo'], PP.params['plotSliceBy']
#print slices 
#nslice = len( slices )

slices = range( 0, len(dfs) )

for ii,i in enumerate(slices):
    print " plotting ", i
    plt.figure( figsize=( 10,10 ) )
    plt.imshow( dfs[i], origin='image', interpolation=PP.params['imageInterpolation'], cmap=PP.params['colorscale'], extent=extent )
    z = zTips[i] - PP.params['moleculeShift' ][2]
    plt.colorbar();
    plt.xlabel(r' Tip_x $\AA$')
    plt.ylabel(r' Tip_y $\AA$')
    plt.title( r"df Tip_z = %2.2f $\AA$" %z  )
    plt.savefig( 'df_%3i.png' %i, bbox_inches='tight' )

    \end{lstlisting}
\end{shadedbox}


